\documentclass[a4paper,10pt]{article}
\usepackage{main}
\usepackage[a4paper, textwidth=170mm, textheight=240mm]{geometry}

\immediate\write18{git describe > version.txt}

\begin{document}
\newlength{\tabcolsepDefault}
\setlength{\tabcolsepDefault}{\tabcolsep}

% color definitions ------------------------------------------------------------

%% \definecolor{lightGray}{HTML}{C0C0C0}
%% \definecolor{Gray}{HTML}{A0A0A0}
\definecolor{tableColor1}{gray}{0.95}
\definecolor{tableColor2}{gray}{0.9}

% general commands -------------------------------------------------------------

\newcommand{\textbi}[1]{\textbf{\textit{#1}}}

\newcommand{\projName}{Jinn's Realism For Farming Simulator}
\newcommand{\projVersion}{v0.7-dev}

\newcommand{\refPageref}[1]{\ref{#1} (p. \pageref{#1})}

%-------------------------------------------------------------------------------

\title{\projName{}\\ \projVersion{}}
\author{Jinnturtle}

\maketitle
\tableofcontents


\section{TODO Before RC}
\begin{itemize}
\item Resolve TODOs in the body of the document.
\item Review probability balance in all tables. Esp. harvest and storage
  penalties.
\item There's a large price jump between tractors from 1990s and 2000s,
  should double-check market info for 1980s tractors and either adjust price or
  reflect the difference in other game system areas e.g. reliability.
\end{itemize}


\section{Mods}

\begin{itemize}
\item Required
  \begin{itemize}
  \item ``Real Loans''
  \item ``Power Tools''
  \end{itemize}
\item Recommended
  \begin{itemize}
    \item Hand-operated and very small (sub 37hp) tools for sowing and planting.
    \item Hand-operated tree planting tools.
    \item Older tractors and equipment from 1990s to at least early 1960s or
      late 1950s (e.g.  Ferguson 35).
    \item A mod that shows fuel per hour consumption.
    \item ``Crop Rotation''.
    \item ``Fruits Adjustment''.
    \item ``Grass Mowing''.
  \end{itemize}
\end{itemize}


\section{Reliability System}
The aim of the reliability system is to make caring for and upgrading equipment
more involved and more rewarding beyond the simple considerations of horse power
and work speed.

Equipment failure checks should be carried out once per month when the equipment
is first used that month. Additional penalties to reliability should be applied
if the equipment was not stored properly (e.g. stored dirty, or left outside).

The older the model of the equipment used, the higher the chance of failure.
Chance of failure should also increase depending on what condition the equipment
was when it was bought.

Roll 1d100, if the result is equal or than less the failure chance, then the
equipment has suffered a failure and needs to be repaired before it can be used.

Major and minor failures can happen independently of each other i.e. both can
happen at the same time or neither. I.e. roll the check once for each.

Reliability penalties from different sources are cumulative. E.g. A tractor
stored in a shed which only has two walls, while dirty, with maintenance 20\%,
paint 0\%, and was bought at used rating B would have additional minor failure
probability of 14\% $(2\% + 3\% + 2\% + 4\% + 3\% = 14\%)$ as per table
\refPageref{tab:equipmentFailRates}.

\TODO{Update rules text with the single roll mechanic explanation and examples.}

\rowcolors{3}{tableColor1}{white}
\begin{table}
\begin{center}
  \caption{Maintenance and Repair Costs and Time}
  \begin{tabular}{lll}
    \toprule
    Failure          & Cost (\%of new) & Downtime  \\
    \midrule
    Minor Breakdown  & 5\%             & $(d10 / 3)$ days\\
    Major Breakdown  & 15\%            & $(4 + d10)$ days \\
    Full Maintenance & $(1d10)$\%      & 1 day\\
    Repaint          & 2\%             & 2 days\\
    \bottomrule
  \end{tabular}
\end{center}
\end{table}

\rowcolors{3}{tableColor1}{white}
\begin{table}
\begin{center}
  \caption{Additional Equipment Failure Rates}
  \begin{tabular}{llcc}
    \toprule
    Source          & Type                  & Minor  & Major \\
    \midrule
    Technology era  & <1960s                & 14\%   & 8\%          \\
    Technology era  & 1960s                 &  9\%   & 6\%          \\
    Technology era  & 1970s                 &  8\%   & 5\%          \\
    Technology era  & 1980s                 &  7\%   & 4\%          \\
    Technology era  & 1990s                 &  6\%   & 3\%          \\
    Technology era  & 2000s                 &  5\%   & 2\%          \\
    Technology era  & 2010/14               &  4\%   & 1\%          \\
    Technology era  & 2015/19               &  3\%   & 1\%          \\
    Technology era  & 2020s                 &  2\%   & 1\%          \\
    Garage          & None (outside)        &  5\%   & 3\%   \\
    Garage          & Roof only             &  3\%   & 2\%   \\
    Garage          & 2 side shed           &  2\%   & 1\%   \\
    Garage          & 3 side shed           &  1\%   & 1\%   \\
    Garage          & Closed garage         &  0\%   & 0\%   \\
    Dirt            & Visibly dirty (>75\%) &  3\%   & 2\%\\
    Maintenance     & <25\%                 &  2\%   & 1\%\\
    Maintenance     & 0\%                   &  3\%   & 2\%\\
    Paint condition & <25\%                 &  1\%   & 0\%\\
    Paint condition & 0\% + outside         &  4\%   & 2\%\\
    Used rating     & A                     &  1\%   & 1\%\\
    Used rating     & B                     &  3\%   & 1\%\\
    Used rating     & C                     &  5\%   & 2\%\\
    Used rating     & D                     & 10\%   & 5\%\\
    \bottomrule
  \end{tabular}
  \label{tab:equipmentFailRates}
\end{center}
\end{table}


\section{Harvest Quality and Storage Spoilage}
Various factors can damage the growing crops and thus have a negative impact on
the amount and quality of the harvest of a crop e.g. drought, pests, plague,
etc.

The quality of the harvested crops in storage may also suffer over time from
e.g. mould, pests, moisture or temperature control errors, etc.

\begin{itemize}
\item For harvest quality, roll 1d100 once per harvest season (e.g. just before
  harvesting a certain crop) against table
  \refPageref{tab:harvestQualityRolls} and note the resulting penalty which
  must later be applied when selling the crops.
\item For crop deterioration in storage events, roll once per month per
  container against table \refPageref{tab:storageDeteriorationRolls} and note
  the resulting penalty.
\end{itemize}

\TODO{Add explanation that penalties are recursively additive and do not
  compound linearly, add one or two calculation examples to minimise confusion.}

\rowcolors{3}{tableColor1}{white}
\begin{table}
\begin{minipage}{0.5\textwidth}
\begin{center}
  \caption{Harvest Quality Roll}
  \rowcolors{3}{tableColor1}{white}
  \begin{tabular}{lcc}
    \toprule
    Score  & Income Penalty \\
    \midrule
    32-100 & $0\%$          \\
    17-31  & $(2d10)\%$     \\
    7-16   & $(20 + 2d10)\%$\\
    2-6    & $(40 + 2d10)\%$\\
    1      & $(60 + 4d10)\%$\\
    \bottomrule
  \end{tabular}
  \label{tab:harvestQualityRolls}
\end{center}
\end{minipage}%
%
\begin{minipage}{0.5\textwidth}
\begin{center}
  \caption{Storage Deterioration Roll}
  \rowcolors{3}{tableColor1}{white}
  \begin{tabular}{lcc}
    \toprule
    Score  & Income Penalty \\
    \midrule
    3-10   & $(1d10)\%$\\
    2      & $(10 + 1d10)\%$\\
    1      & $(20 + 1d10)\%$\\
    \bottomrule
  \end{tabular}
  \label{tab:storageDeteriorationRolls}
\end{center}
\end{minipage}
\end{table}


\section{Acquiring Equipment}
\subsection{Buying}
New equipment is always available to buy, but the price may fluctuate a bit.
Used equipment may be harder to find, especially in the desired condition.

The steps to follow when buying equipment are:
\begin{enumerate}
\item (Only if buying used.) Roll 1d100 against table
  \refPageref{tab:usedEquipmentOffers} to determine how many offers there are
  available on the market.
\item (Only if buying used.) Then roll 1d10 against the table
  \refPageref{tab:usedEquipmentOfferRatings} to determine the condition of
  equipment on sale for each offer.
\item Refer to \refPageref{tab:equipmentPrices} to determine the base price.
\item Refer to \refPageref{tab:equipmentPriceModifiers} to determine the price
  modifier.
\item Calculate the final price using formula $P_b \times M_c$ where $P_b$ is
  the base price and $M_c$ is the modifier. You can see some examples in table
  \refPageref{tab:equipmentPriceCalculationExamples}.
\end{enumerate}

The results are valid only for the month they are rolled in, so if you'd like to
try again you can do so next month, but if you do like the results make sure to
grab what's on offer before the end of the month. The roll can be made once per
model of equipment in question, so searching for a similar model in case initial
results are not satisfactory is a valid strategy, this however may result in
more opportunities than plausible if the user has many similar mods installed so
the user is encouraged to limit the number of attempts to what looks plausible
given current market conditions.


\subsection{Hire}

Modern equipment hire (models from the last 10 years) is quite straightforward
and readily available, the per week is 1\% price of new reduce by 10\% if hiring
for more than 8 weeks and by 20\% if hiring for more than 16 weeks. Apply (1d10
- 1d10)\% modifier to the final price if you'd like to simulate a less
predictable market.

Older equipment is probably not for hire from the bigger companies online, but
if you have neighbours they may be able to help you. Roll against 30\% chance if
you're looking for something made in the last 11 to 30 years, and against 10\%
chance if you're looking for something older still. The hire rate is 1\% of used
category ``A'' price per week.

\subsection{Equipment Wear Categories}

The general wear and tear condition of equipment is important when performing
reliability check calculations and when buying equipment. The the general state
of equipment wear can be categorised as follows.
\begin{description}
\item[N] Equipment was bought brand new, has spent less than 5 years on the
  farm and has accumulated less than 100 hours.
\item[A] The machine has seen little work with the previous owner or,
  especially in case of historical equipment, has been fully refurbished to near
  factory-new state. (A modern tractor would probably have accumulated between
  100 and 1000 hours).
\item[B] The machine has seen regular work for some time now, but is well
  looked after, no obvious signs of rust, etc. (A modern tractor would probably
  have accumulated between 1000 and 6000 hours).
\item[C] The machine has seen considerable amount of work for many years, or
  has only seen basic care for a few years, may have some visible signs of wear.
  (A modern tractor would probably have accumulated between 6000 and 12000
  hours.)
\item[D] The machine has been used and abused for many years, it may still be
  fit for work, but maintenance issues should be expected. This piece of
  equipment wold probably not look out of place in a scrap yard.
\end{description}

\begin{table}
  \caption{Equipment Prices}
  \rowcolors{3}{tableColor1}{white}
  \begin{minipage}{0.5\textwidth}
  \begin{center}
  \begin{tabular}{lcccc}
    \toprule
    \multicolumn{5}{c}{Tractors (Price in \pounds per horsepower)}\\
    Era     & <200hp & <100hp & <70hp & <50hp\\
    \midrule
    1950s*  &        & 210    & 165   & 150\\
    1960s*  &        & 220    & 200   & 220\\
    1970s*  &        & 230    & 215   &\\
    1980s*  &        & 235    & 270   &\\
    1990s   & 250    & 270    &       &\\
    2000s   & 350    & 380    &       &\\
    2010/14 & 445    & 455    &       &\\
    2015/19 & 480    &        &       &\\
    \midrule
    \rowcolor{white}\multicolumn{5}{l}{* 2WD prices. Add (20 + 2d10)\% for MFWD.}\\
    \bottomrule
  \end{tabular}
  \end{center}
  \end{minipage}%
%
  \rowcolors{3}{tableColor1}{white}
  \begin{minipage}{0.5\textwidth}
  \begin{center}
  \begin{tabular}{lcc}
    \toprule 
    \multicolumn{3}{c}{Tipper Trailers (Price in \pounds)}\\
    Capacity  & Per Tonne & Per $m^3$\\
    \midrule
    $<10m^3$  & 800       & 1,500\\
    $10-30m^3$& 650       & 1,250\\
    $>30m^3$  & 600       & 700\\
    \bottomrule
  \end{tabular}
  \end{center}
  \end{minipage}
  \label{tab:equipmentPrices}
\end{table}

\begin{table}
\begin{center}
  \caption{Equipment Price Modifiers}
  \rowcolors{3}{tableColor1}{white}
  \begin{tabular}{cc}
    \toprule
    Condition & Modifier\\
    \midrule
    N         & $(100 + 1d10 - 1d10)\%$\\
    A         & $(70 + 1d10)\%$\\
    B         & $(70 - 2d10)\%$\\
    C         & $(50 - 1d10)\%$\\
    D         & $(35 - 1d10)\%$\\
    \bottomrule
  \end{tabular}
  \label{tab:equipmentPriceModifiers}
\end{center}
\end{table}

\begin{table}
  \caption{Equipment Price Calculation Examples}
  \rowcolors{5}{tableColor1}{white}
  \begin{center}
  \begin{tabular}{ll}
    Assuming base price $(P_b)$ of \pounds{}20,000.\\
    \\
    \toprule
    Example &
    Calculation
    \\
    \midrule
    Pristine machine with lucky $M_c$ roll of $(100 + 3 - 9)\%$ &
    $P_b \times M_c = \pounds{}20,000 \times 94\% = \pounds{}18,800$
    \\
    Category A used machine with $M_c$ roll of $(70 + 6)\%$ &
    $P_b \times M_c = \pounds{}20,000 \times 76\% = \pounds{}14,800$
    \\
    Category C used machine, luckiest roll $(50 - 10)\%$&
    $P_b \times M_c = \pounds{}20,000 \times 40\% = \pounds{}8,000$
    \\
    Category C used machine, average roll $(50 - 5)\%$&
    $P_b \times M_c = \pounds{}20,000 \times 45\% = \pounds{}9,000$
    \\
    Category C used machine, most unlucky roll $(50 - 1)\%$&
    $P_b \times M_c = \pounds{}20,000 \times 49\% = \pounds{}9,800$
    \\
    \bottomrule
  \end{tabular}
  \end{center}
  \label{tab:equipmentPriceCalculationExamples}
\end{table}

\begin{table}
\begin{minipage}{0.5\textwidth}
\begin{center}
  \caption{Used Equipment Availability}
  \rowcolors{3}{tableColor1}{white}
  \begin{tabular}{lc}
    \toprule
    Score  & Offers Available\\
    \midrule
    1-25   & 0\\
    26-50  & 1\\
    51-75  & 2\\
    76-100 & 3\\
    \bottomrule
  \end{tabular}
  \label{tab:usedEquipmentOffers}
\end{center}
\end{minipage}%
%
\begin{minipage}{0.5\textwidth}
\begin{center}
  \caption{Used Equipment Condition}
  \rowcolors{3}{tableColor1}{white}
  \begin{tabular}{lc}
    \toprule
    Score  & Condition\\
    \midrule
    1-2    & A\\
    3-5    & B\\
    6-8    & C\\
    9-10   & D\\
    \bottomrule
  \end{tabular}
  \label{tab:usedEquipmentOfferRatings}
\end{center}
\end{minipage}
\end{table}


\section{Additional Expenses and Time}
\begin{description}
\item[Fuel] Suggested price \pounds{}1.50 (avg. price in UK in 2023-08). Mods
  providing varied size fuel containers for purchase as ``pallettes'' are quite
  helpful, at least in the beginning of the game.
\item[Personal expenses] Roll 4d100 at end of each month. This represents
  personal expenses like food, clothing, home utilities, etc. Feel free to
  modify this roll in accordance to role-play, e.g. one dice can be subtracted
  to represent replacement of big portion of the diet with produce grown on
  farm, 2 dice to represent home grown diet and more conservative lifestyle,
  etc.
\item[Travel Distance] Consider the fuel needed for a trip to town, market, etc.
  Suggested distance is 25km (one way). This is to add some depth to the need
  for trailers and efficient tractors or trucks for delivery of goods. A mod
  that shows fuel consumption rate is especially helpful for this.
\end{description}
\end{document}
